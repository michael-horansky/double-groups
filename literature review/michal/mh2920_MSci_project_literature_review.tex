\documentclass[journal]{Imperial_lab_report}

\usepackage{url}
\usepackage{physics}
\usepackage{siunitx}
\usepackage{graphicx}
\usepackage{caption}
% correct bad hyphenation here
\hyphenation{op-tical net-works semi-conduc-tor}


\begin{document}

\title{Double Groups and Quantum Dots - Literature Review}%
\author{Michal Horansky}


\markboth{M. HORANSKY}%
{Horansky \MakeLowercase{\textit{et al.}}:}

\maketitle

\begin{abstract}

In this project, we investigate the photonic properties of quantum dots employing an approach analogous to standard procedures in studies of the topic. Experimental data from polarisation resolved photoluminiscence spectroscopy are investigated using the theory of exciton complexes and group theory. The major features of the spectral diagrams can be labelled by exciton state transitions immediately by considering the properties of states of these exciton complexes. These major features also exhibit splitting caused by fine-structure spin interactions. This splitting is modelled using double groups, and agreement with polarisation and intensity of each resolvable peak is assessed. Predictions for dark and unresolved peaks are stated. Possible symmetry elevations in the quantum dot are discussed with consideration for their effects on the observed spectra.

\end{abstract}


\section{Introduction}
\IEEEPARstart{D}{ouble} groups are awesome

what are qds?

what does the data mean

	mmm spectra -> double groups
	
what are double groups

goal: deduce the symmetries of the qd based on the spectra
(growth mode BUT symmetry elevations)

what is symmetry elevation

\section{Background Material - BY THURSDAY EVENING}

why polarisation in data? -> group theory prediciton
why temperature data> -> to separate heavy-like and light-like

classify sources

classify steps

\subsection{Exciton analysis}
lmao

\subsection{Potential analysis}
or join this with exciton analysis

\subsection{Symmetry analysis}

\section{Summary}
anything

%\begin{thebibliography}{1}

%\bibitem{microwave}
%Steer, M. B. (2010) \textit{Microwave and RF design: A systems approach}. Beta edn. North Carolina State University: SciTech Publishing Inc.

%\end{thebibliography}


% Can use something like this to put references on a page
% by themselves when using endfloat and the captionsoff option.
\ifCLASSOPTIONcaptionsoff
  \newpage
\fi

\appendix[Feedback for the 1st cycle two-page report]
"Decent in all categories - lab book had good recording of what you did. Fig captions should tell reader what to take away, not just what's in the figure. Good description of method, numerical result and error all there - some attempt at error analysis."

% trigger a \newpage just before the given reference
% number - used to balance the columns on the last page
% adjust value as needed - may need to be readjusted if
% the document is modified later
%\IEEEtriggeratref{8}
% The "triggered" command can be changed if desired:
%\IEEEtriggercmd{\enlargethispage{-5in}}

% references section

% can use a bibliography generated by BibTeX as a .bbl file
% BibTeX documentation can be easily obtained at:
% http://mirror.ctan.org/biblio/bibtex/contrib/doc/
% The IEEEtran BibTeX style support page is at:
% http://www.michaelshell.org/tex/ieeetran/bibtex/
%\bibliographystyle{IEEEtran}
% argument is your BibTeX string definitions and bibliography database(s)
%\bibliography{IEEEabrv,../bib/paper}
%
% <OR> manually copy in the resultant .bbl file
% set second argument of \begin to the number of references
% (used to reserve space for the reference number labels box)
\bibliographystyle{ieeetran} 
% use the following line to create a bibligraphy based on a .bib bibtex file (change to your filename)
%\bibliography{filename}





% that's all folks
\end{document}


