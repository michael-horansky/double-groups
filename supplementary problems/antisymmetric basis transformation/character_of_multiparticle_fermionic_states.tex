\documentclass[12pt]{article}
\usepackage{stdheadstart}
\usepackage{xargs}
\usepackage{physics}
\usepackage{amsmath,amssymb}
\insheadstart{images/}
\newcommand{\sgn}{\text{sgn}}

\newtheorem{full_coupling_hilbert_space}
{Theorem}

\begin{document}

	\title{The character of a degenerate $j$-eigenstate occupied by an ensemble of fermions under simultaneous rotation.}	
	\maketitle
	
	\abstract{An eigenstate of angular momentum (i.e. a $j$-eigenstate) is $(2j+1)$-fold degenerate. For a single particle, choosing the eigenstates of $j_z$ as a basis allows us to determine the character of a single particle under rotation as the trace of the corresponding Wigner d-matrix $d^j_{m'm}$. Constructing multi-particle states as tensor products of $j_z$-eigenstates, we equate the character of the Slater determinant describing $k$ fermions occupying the same $j$-eigenstates to the sum of the rank-$k$ principal minors of $d^j$. Stating the relation between the rotation character and the coefficients of the characteristic polynomial of $d^j$, we use the Faddeev-Le Verrier algorithm (FLV) to write down an expression for the multi-fermionic rotation character as a function of traces of $d^j$ and its exponentiations. From the unitarity of $d^j$, we prove that the rotation character of an ensemble of $k$ fermions with angular momentum $j$ equals the rotation character of an ensemble of $2j+1-k$ fermions of equal angular momentum. We prove that the rotation character can be uniquely decomposed into a sum of rotation characters of single particles with arbitrary angular momenta, utilising the orthogonality relations of Chebyshev polynomials of the second kind.}
	
	\section{Deriving the rotation character of multi-fermionic states}
	\subsection{Rotation character of distinguishable particle states}
	Let us denote an eigenstate of angular momentum $j$ and its projection $m$ onto an arbitrary axis (conventionally chosen to be the $z$-axis) as $\ket{j, m}$. It can be proven (TODO citation) that the set of states
	$$\ket{j, -j}, \ket{j, -j+1} \dots \ket{j, j-1}, \ket{j, j}$$
	forms an orthonormal basis of the Hilbert space of states with angular momentum $j$. Then, the character of a $j$-eigenstate under rotation $\hat{R}(\theta)$ is the sum of characters of the basis vectors under the same rotation:
	\begin{equation} \label{rot_char_of_single_p}
	\chi^j(\theta)=\mel{j}{\hat{R}(\theta)}{j}=\sum_{m=-j}^j \mel{j, m}{\hat{R}(\theta)}{j, m}
	\end{equation}
	Note that here the rotation is described by a single scalar parameter; indeed, it can be proven (TODO citation) that the rotation character is given only by the rotation angle, not the rotation axis, and hence we need only consider rotations about an arbitrarily chosen axis. Fixing the rotation axis as the $z$-axis, we can immediately equate the matrix element in Eq. \ref{rot_char_of_single_p} to the Wigner d-matrix:
	\begin{eqnarray}
	\mel{j, m'}{\hat{R}(\theta)}{j, m} &=& d^j_{m'm}(\theta)\\
	\chi^j(\theta)&=&\Tr d^j(\theta) = \frac{\sin{(2j+1)\frac{\theta}{2}}}{\sin{\frac{\theta}{2}}} \label{single_particle_character}
	\end{eqnarray}
	Now, consider an ensemble of two distinguishable particles $A$ and $B$, with angular momenta $j_A, j_B$. The Hilbert space of this ensemble is the tensor product of two single-particle Hilbert spaces, and as such it is spanned by the vectors
	\begin{equation} \label{two_particle_basis}
	\ket{j_A, m_A}_A\otimes\ket{j_B, m_B}_B, \quad m_A = -j_A \dots j_A, \quad m_B = -j_B \dots j_B
	\end{equation}
	A \textit{simultaneous rotation} can then be constructed as the tensor product of the corresponding rotation operators:
	$$\hat{R}(\theta)=\hat{R}_A(\theta)\otimes\hat{R}_B(\theta)$$
	Then, the character of this rotation for the vector basis in Eq. \ref{two_particle_basis} is equal to the product of the characters of constituent particles:
	\begin{equation}
	\chi^{j,j'}(\theta)=\chi^j(\theta)\chi^{j'}(\theta)
	\end{equation}
	This relation can be easily generalised--for an ensemble of $k$ distinguishable particles, each with angular momentum $j_i$, the simultaneous rotation character of the vector basis is $\prod_{i}\chi^{j_i}(\theta)$.
	
	Note that for the case of zero particles--that is, the vacuum state $\ket{\emptyset}$--the character is trivially $1$, since the vacuum state is invariant under rotation, and we have
	\begin{equation} \label{vacuum_character}
	\chi^\emptyset(\theta)=\mel{\emptyset}{\hat{R}(\theta)}{\emptyset}=\braket{\emptyset}{\emptyset}=1
	\end{equation}
	
	\subsection{Rotation character of multiple fermions occupying a single $j$-eigenstate}
	Now, suppose we have $k$ fermions occupying a single $j$-eigenstate (i.e. their angular momenta are all equal to $j$). Suppose the angular momentum projection of the $i$-th particle is $m_i$, with $m_i\neq m_j$ for $i\neq j$. We shall denote the corresponding vector as $\ket{j; m_1, m_2\dots m_k}$. The character of this state under rotation can be treated as if the fermions were distinguishable particles, since they occupy different eigenstates of $m$, which are orthogonal to each other. Hence, the character of this state under rotation is
	\begin{equation}
	\mel{j; m_1, m_2\dots m_k}{\hat{R}(\theta)}{j; m_1, m_2\dots m_k}=\prod_i \mel{j, m_i}{\hat{R}(\theta)}{j, m_i}
	\end{equation}
	This vector is an eigenstate of $m_i$, but a physical system cannot exist purely in such a state, since for an ensemble of fermions, we demand that their quantum state is antisymmetric under the exchange of any two fermions.
	
	Let us denote a suitable quantum state for an ensemble of $k$ fermions with angular momentum $j$ which occupy a set of $m$-eigenstates $m_i$ as $\ket{b^j_k; m_1, m_2\dots m_k}$. Let us express this state as a superposition of the $m_i$-eigenstates:
	\begin{equation}
	\ket{b^j_k; m_1, m_2\dots m_k}=\sum_{P^k_x}c_x\ket{j; P^k_x\left(m_1, m_2\dots m_k\right)}
	\end{equation}
	 where $P^k$ is a permutation of $k$ elements and the sum is over all such permutations. Then, permuting the fermions with an arbitrary permutation $P^k_y$ yields
	 \begin{equation}
	 \hat{P}^k_y\ket{b^j_k; m_1, m_2\dots m_k}=\sum_{P^k_x}c_x\ket{j; (P^k_yP^k_x)\left(m_1, m_2\dots m_k\right)}=\sgn(P^k_y)\ket{b^j_k; m_1, m_2\dots m_k}
	 \end{equation}
	where $\sgn(P^k)$ is the sign function, equal to $1$ if $P^k$ is an even permutation and $-1$ if $P_k$ is an odd permutation.
	
	Let us choose $P^k_y=\left(P^k_{x'}\right)^{-1}$. By comparing the coefficients of the $m_i$-eigenstates, we obtain the relation
	$$c_0=\sgn(P^k_{x'})c_{x'}$$
	where $P^k_0$ is the identity element (empty permutation). 	Hence, fixinging $c_0$, all remaining coefficients are given as
	$$c_x=\sgn(P^k_x)c_0$$
	The value of $c_0$ is determined by demaning the state to be normalized, up to a complex phase; we choose it to be $(k!)^{-1/2}$. Hence, the fermionic quantum state is
	\begin{equation} \label{slater_determinant}
	\ket{b^j_k; m_1, m_2\dots m_k}=\frac{1}{\sqrt{k!}}\sum_{P^k_x}\sgn(P^k_x)\ket{j; P^k_x\left(m_1, m_2\dots m_k\right)}
	\end{equation}
	This is of course simply the Slater determinant constructed from the $m$-eigenstates for each occupied value of $m$.
	
	\subsection{The rotation character of the multi-fermionic Slater determinant}
	Under simultaneous rotation, the character of a single multi-fermionic quantum state is
	$$\mel{b^j_k; m_1, m_2\dots m_k}{\hat{R}(\theta)}{b^j_k; m_1, m_2\dots m_k}$$
	Expanding this using Eq. \ref{slater_determinant}, we obtain
	\begin{align*}
	\mel{b^j_k; m_1, m_2\dots m_k}{\hat{R}(\theta)}{b^j_k; m_1, m_2\dots m_k} =\\
	\frac{1}{k!}\sum_{x}\sum_{y} \sgn(P^k_x)\sgn(P^k_y)\mel{j; P^k_x\left(m_1, m_2\dots m_k\right)}{\hat{R}(\theta)}{j; P^k_y\left(m_1, m_2\dots m_k\right)}
	\end{align*}
	Since the matrix element of a $m_i$-eigenstate is invariant under a simultaneous permutation of both the bra and the ket, as the corresponding pairs of $m'_i, m_i$ remain unchanged, we shall simultanously permute each term by $(P^k_x)^{-1}$.
	\begin{align*}
	\mel{b^j_k; m_1, m_2\dots m_k}{\hat{R}(\theta)}{b^j_k; m_1, m_2\dots m_k} =\\
	\frac{1}{k!}\sum_{x}\sum_{y} \sgn(P^k_x)\sgn(P^k_y)\mel{j; m_1, m_2\dots m_k}{\hat{R}(\theta)}{j; \left(\left(P^k_x\right)^{-1}P^k_y\right)\left(m_1, m_2\dots m_k\right)}
	\end{align*}
	Let us denote $P^k_z=\left(P^k_x\right)^{-1}P^k_y$. As we have $\sgn(\left(P^k_x\right)^{-1})=\sgn(P^k_x)$, we obtain
	\begin{align*}
	\mel{b^j_k; m_1, m_2\dots m_k}{\hat{R}(\theta)}{b^j_k; m_1, m_2\dots m_k} =\\
	\frac{1}{k!}\sum_{x}\sum_{y} \sgn(P^k_z)\mel{j; m_1, m_2\dots m_k}{\hat{R}(\theta)}{j; P^k_z\left(m_1, m_2\dots m_k\right)}
	\end{align*}
	Note that the elements of $P^k$ form a group, and hence we have
	$$P^k_A P^k_B = P^k_A P^k_C \iff P^k_B = P^k_C$$
	Hence acting on each element of the set with a single element $\left(P^k_x\right)^{-1}$ leaves the set invariant, as each permutation is present once still. Hence the sum over $y$ can be rewritten as a sum over $z$. This removes the dependence on $x$, and the sum over $x$ results in a factor equal to the order of the group $P^k$, which is $\abs{P^k}=k!$. Hence
	\begin{align*}
	\mel{b^j_k; m_1, m_2\dots m_k}{\hat{R}(\theta)}{b^j_k; m_1, m_2\dots m_k}\\
	=\sum_{z} \sgn(P^k_z)\mel{j; m_1, m_2\dots m_k}{\hat{R}(\theta)}{j; P^k_z\left(m_1, m_2\dots m_k\right)}\\
	=\sum_{z} \sgn(P^k_z) \prod_{i=1}^k d^j_{m_i, m'_i}(\theta)
	\end{align*}
	where $m'_i$ is the $i$-th element of the array of $m_i$ permuted by $P^k_z$. We recognise the expression as the determinant of a matrix comprised of columns and rows of $d^j$ with indices present in the set $\{m_1, m_2\dots m_k\}$, i.e. the principal minor of $d^j$ determined by this set.
	
	This is the character of a single antisymmetric basis vector of the multi-fermionic ensemble. In the case where we do not know which values of $m$ are occupied, we calculate the total character of the ensemble as a sum over the characters of all antisymmetric basis vectors, i.e. the sum of all principal minors of $d^j$ of rank $k$. However, we know (TODO citation) that the characteristic polynomial of a matrix $M$ has the form:
	\begin{equation} \label{principal_minor_polynomial}
	p(\lambda)=\det(\lambda I - M) = \sum_{k=0}^{\dim M}\lambda^{\dim M-k}(-1)^k T_k
	\end{equation}
	where $T_k$ is the sum of all principal minors of $M$ of rank $k$. Therefore, applying Eq. \ref{principal_minor_polynomial} to the Wigner d-matrix, we find that the characteristic polynomial of $d^j_{m'm}$ is equal to
	\begin{equation} \label{wigner_characteristic_polynomial}
	p^{\left(d^j(\theta)\right)}(\lambda) = \sum_{k=0}^{2j+1}\lambda^{k}(-1)^{2j+1-k} \chi^j_{2j+1-k}(\theta)
	\end{equation}
	where $\chi^j_a$ is the character of an ensemble of $a$ fermions with angular momentum $j$ under simultaneous rotation by $\theta$, where the degeneracy between their $m$-values is not broken by measurement.
	
	\subsection{Applying the Faddeev-Le Verrier algorithm}
	The Faddeev-Le Verrier algorithm (FLV) allows us to find an expression for the coefficients of the characteristic polynomial of a matrix $M$. Suppose the polynomial is
	$$p(\lambda)=\sum_{k=0}^n c_k \lambda^k$$
	where $n=\dim M$. Then
	\begin{equation} \label{Faddeev-LeVerrier}
	c_{n-m} = -\frac{1}{m}\sum_{k=1}^m c_{n-m+k} \Tr\left(M^k\right)
	\end{equation}
	Applying Eq. \ref{Faddeev-LeVerrier} to $d^j_{m'm}(\theta)$, identifying $n=2j+1$, and equating the powers of $\lambda$ to Eq. \ref{wigner_characteristic_polynomial}, we obtain
	\begin{equation} \label{FLV_with_matrix_powers}
	\chi^j_k(\theta)=-\frac{1}{k}\sum_{a=1}^k (-1)^{a}\chi^j_{k-a}(\theta)\Tr\left(\left(d^j(\theta)\right)^a\right)
	\end{equation}
	where we can identify $\chi^j_0$ as the character of the vacuum state under rotation, which, as per Eq. \ref{vacuum_character}, is trivially equal to unity; hence we have $\chi^j_0(\theta)=1$.
	
	Eq. \ref{FLV_with_matrix_powers} can be simplified further by interpreting $d^j(\theta)$ as a matrix representation of an element of $SU(2)$ which corresponds to rotation about the $z$-axis by an angle $\theta$. Applying said rotation $a$ times (that is, taking the $a$-th power of the matrix) is equivalent to a single rotation by an angle equal to $a\theta$. Therefore we have
	\begin{equation}
	\Tr\left(\left(d^j(\theta)\right)^a\right) = \Tr d^j(a\theta)
	\end{equation}
	which allows us to rewrite Eq. \ref{FLV_with_matrix_powers} as
	\begin{equation} \label{ensemble_character_recursive_relation}
	\chi^j_k(\theta)=-\frac{1}{k}\sum_{a=1}^k (-1)^{a}\chi^j_{k-a}(\theta)\Tr d^j(a\theta)
	\end{equation}
	
	\subsection{The symmetries of $\chi^j_k$}
	The Wigner d-matrices $d^j_{m'm}(\theta), \theta\in (0,4\pi)$ form a subset of the Wigner $\mathcal{D}$-matrices $\mathcal{D}^j_{m'm}(\alpha, \beta, \theta)$, which form irreducible representations of the group $SU(2)$. As such, $d^j_{m'm}(\theta)$ is a unitary matrix with determinant one. This allows us to derive an important symmetry regarding the coefficients of its characteristic polynomial, and hence the characters $\chi^j_k$:
	\begin{eqnarray}
	\lambda^{2j+1}p^{\left(d^j(\theta)\right)}\left(\frac{1}{\lambda}\right) &=& \lambda^{2j+1}\det(\frac{1}{\lambda}I-d^j(\theta))\\
	&=& \det(I-\lambda d^j(\theta))\\
	&=& \det(\left(d^j\right)^Td^j(\theta)-\lambda d^j(\theta))\\
	&=& \det(\left(d^j\right)^T-\lambda I)\det(d^j(\theta))\\
	&=& \det(\left(d^j-\lambda I\right)^T)\\
	&=& (-1)^{2j+1}p^{\left(d^j(\theta)\right)}(\lambda)
	\end{eqnarray}
	Expressing the characteristic polynomial using Eq. \ref{wigner_characteristic_polynomial} and comparing equal powers of $\lambda$, we retrieve the condition
	\begin{equation} \label{flip_k_by_two_j_plus_one}
	\chi^j_k(\theta)=\chi^j_{2j+1-k}(\theta)
	\end{equation}
	Therefore, changing the number of occupied states in a $j$-eigenstate to the number of unoccupied states in the same $j$-eigenstate leaves the character under simultaneous rotation invariant.
	
	Another important set of symmetries is that for the argument $\theta$. Firstly, we note that, as discussed before, the character of a rotation does not depend on the axis of rotation. If we therefore invert the axis, the character remains invariant. However, inverting the rotation axis is equivalent to changing the sign of the rotation angle. Therefore we obtain
	\begin{equation} \label{flip_angle_sign}
	\chi^j_k(-\theta)=\chi^j_k(\theta)
	\end{equation}
	Secondly, let us identify $\Tr d^j(\phi)=\chi^j_1(\phi)$ in Eq. \ref{ensemble_character_recursive_relation}:
	\begin{equation}
	\chi^j_k(\theta)=-\frac{1}{k}\sum_{a=1}^k (-1)^{a}\chi^j_{k-a}(\theta)\chi^j_1(a\theta)
	\end{equation}
	We shall prove the following relation:
	\begin{equation} \label{character_sign_under_two_pi}
	\chi^j_k(2\pi+\theta)=(-1)^{2jk}\chi^j_k(\theta)
	\end{equation}
	A corollary of this relation is that
	\begin{equation} \label{character_sign_under_integer_two_pi}
	\chi^j_k(2a\pi+\theta)=(-1)^{2ajk}\chi^j_k(\theta)
	\end{equation}
	for integer $a$.
	
	Let us prove Eq. \ref{character_sign_under_two_pi} by induction. Firstly, from Eq. \ref{single_particle_character}
	\begin{equation}
	\chi^j_1(2\pi+\theta)=\frac{\sin((2j+1)\pi+(2j+1)\frac{\theta}{2})}{\sin(\pi +\frac{\theta}{2})}=(-1)^{2j}\chi^j_k(\theta)
	\end{equation}
	where we have used the identity $\sin(a\pi+x)=(-1)^a\sin{x}$ for integer $a$. Also, using Eq. \ref{vacuum_character}, we see that Eq. \ref{character_sign_under_two_pi} trivially holds for empty states:
	\begin{equation}
	\chi^j_0(2\pi+\theta)=1=(-1)^0\chi^j_0(\theta)
	\end{equation}
	
	Now, assuming Eq. \ref{character_sign_under_two_pi} holds for states with $a$ fermions with $a=0,1\dots k-1$, let us consider shifting the argument of an ensemble with $k$ fermions:
	\begin{align}
	\chi^j_k(2\pi+\theta)&=-\frac{1}{k}\sum_{a=1}^k (-1)^{a}\chi^j_{k-a}(2\pi+\theta)\chi^j_1(2a\pi+a\theta)\nonumber\\
	&=-\frac{1}{k}\sum_{a=1}^k (-1)^{a}(-1)^{2j(k-a)}\chi^j_{k-a}(\theta)(-1)^{2ja}\chi^j_1(2a\pi+a\theta)\nonumber\\
	&=(-1)^{2jk}\chi^j_k(\theta) \label{shift_by_two_pi}
	\end{align}
	which finishes the proof. Hence we have proved Eq. \ref{character_sign_under_two_pi} and its corollary Eq. \ref{character_sign_under_integer_two_pi}, from which we also have trivially that all the ensemble characters are $4\pi$-periodic, i.e.
	\begin{equation} \label{character_four_pi_periodic}
	\chi^j_k(4\pi+\theta)=\chi^j_k(\theta)
	\end{equation}
	
	\section{Multi-fermionic rotation characters as Gaussian binomial coefficients}
	\subsection{The eigenvalues of the Wigner d-matrices}
	Consider an irrep of $SU(2)$ of dimension $2j+1$, denoted $\Gamma_j$. By the spectral theorem, we can choose a basis $\ket{d, i}$ which diagonalizes $\Gamma_j$, in which we have
	\begin{equation}
	D^{(\Gamma_j)}\left(\hat{R}(\theta)\right)=\mqty(\dmat{f_1(\theta),f_2(\theta),\ddots,f_{2j+1}(\theta)})
	\end{equation}
	so that
	\begin{eqnarray}
	\hat{R}(\theta)\ket{b,i}=f_i(\theta)\ket{b, i}
	\end{eqnarray}
	where $\ket{b,i}$ is a linear superposition of the $j$-eigenstates with $j_z=-j\dots j$ obtained by a unitary transformation (so that its norm is one). Note that $f_i(\theta)$ is the $i$-th eigenvalue of $D^{(\Gamma_j)}\left(\hat{R}(\theta)\right)$, which is a general rotation by $\theta$.
	
	Because $SU(2)$ is unitary, we have
	\begin{equation}
	\mel{b,i}{\hat{R}(\theta)^\dagger \hat{R}(\theta)}{b,i}=\mel{b,i}{f_i(\theta)^*f_i(\theta)}{b,i}=f_i(\theta)^*f_i(\theta)=1
	\end{equation}
	Therefore $f_i(\theta)=\exp(ig_i(\theta))$, where $g_i(\theta)$ is real.
	
	Consider now an $a$-fold rotation by $\theta$, which is represented by the matrix
	$$\left[D^{(\Gamma_j)}\left(\hat{R}(\theta)\right)\right]^a=D^{(\Gamma_j)}\left(\hat{R}(a\theta)\right)$$
	The action of this rotation on an eigenstate $\ket{b,i}$ is then
	\begin{equation}
	\hat{R}(a\theta)\ket{b,i} = f_i(a\theta)\ket{b,i} = f_i(\theta)^a\ket{b,i}
	\end{equation}
	Substituing in the expression for $f_i$ in terms of $g_i$ shows that $g_i(\theta)$ is a linear function: $g_i(\theta) = G_i\theta$, where $G_i$ is a real constant, and the sum of $f_i(\theta)$ is then the Fourier series decomposition of the character $\chi^j_1(\theta)$.
	
	We identify Eq. \ref{single_particle_character} as the Dirichlet kernel, which can be expressed as a sum:
	\begin{equation} \label{dirichlet_kernel_sum}
	\chi^j_{k=1}(\theta)=\Tr D^{(\Gamma_j)}\left(\hat{R}(\theta)\right)=\frac{\sin((2j+1)\frac{\theta}{2})}{\sin{\frac{\theta}{2}}}=\sum_{c=-j}^j e^{ic\theta}
	\end{equation}
	Since $e^{ic\theta}$, $c\in \mathbb{R}$ form an orthonormal basis to $f:\mathbb{R}\to\mathbb{C}$, the decomposition of the trace into its Fourier series is unique, and hence we can identify $G_i=-j-1+i$, and therefore
	\begin{equation} \label{su2_eigenvalues}
	e^{ic\theta}, c=-j,-j+1\dots j-1, j \qq{are the eigenvalues of} D^{(\Gamma_j)}\left(\hat{R}(\theta)\right)
	\end{equation}
	Since the Wigner d-matrices form a subgroup of $SU(2)$, naturally their eigenvalues are given by Eq. \ref{su2_eigenvalues}.
	
	\subsection{Expanding the characteristic polynomial of the Wigner d-matrices}
	By Eq. \ref{wigner_characteristic_polynomial}, the relevant characteristic polynomial of $d^j(\theta)$ is monic. Therefore, we can construct it from its roots (i.e. eigenvalues of the representation) like so:
	\begin{equation} \label{wigner_simple_characteristic_polynomial_with_roots}
	p(\lambda)=\prod_{c=-j}^j\left(\lambda-e^{ic\theta}\right)
	\end{equation}
	To expand this polynomial, we can utilise the $q$-binomial theorem (TODO cite this), which states that
	\begin{equation}
	\prod_{k=0}^{n-1}\left(1+q^kt\right)=\sum_{k=0}^{n}q^{k(k-1)/2}{n\choose k}_q t^k
	\end{equation}
	where ${n\choose k}_q$ is the Gaussian binomial coefficient defined as
	\begin{equation} \label{gaussian_binomial_coefficient}
	{n\choose k}_q = \frac{\left(1-q^n\right)\left(1-q^{n-1}\right)\dots \left(1-q^{n-k+1}\right)}{\left(1-q\right)\left(1-q^2\right)\dots \left(1-q^k\right)}
	\end{equation}
	We bring Eq. \ref{wigner_simple_characteristic_polynomial_with_roots} into the same form:
	\begin{eqnarray}
	p(\lambda)&=&\prod_{c=-j}^j\lambda\left(1-e^{ic\theta}\lambda^{-1}\right)\\
	&=&\lambda^{2j+1}\prod_{k=0}^{2j}\left(1-\left(e^{i\theta}\right)^k e^{-ij\theta}\lambda^{-1}\right)
	\end{eqnarray}
	We identify $n=2j+1, q=e^{i\theta}, t=-e^{-ij\theta}\lambda^{-1}$ and apply the $q$-binomial theorem:
	\begin{eqnarray}
	p(\lambda)&=&\lambda^{2j+1}\sum_{k=0}^{2j+1}e^{i(k(k-1)/2)\theta}{2j+1\choose k}_{e^{i\theta}}(-1)^k e^{-ijk\theta}\lambda^{-k}\\
	&=&\sum_{k=0}^{2j+1}\lambda^{2j+1-k}(-1)^k e^{i(k(k-1)/2-jk)\theta}{2j+1\choose k}_{e^{i\theta}}
	\end{eqnarray}
	Comparing this to Eq. \ref{wigner_characteristic_polynomial}, we obtain
	\begin{equation} \label{multifermionic_character_gaussian_binomial}
	\chi^j_k(\theta) = \exp(-i\frac{k(2j+1-k)}{2}\theta) {2j+1\choose k}_{e^{i\theta}}
	\end{equation}
	We can express Eq. \ref{multifermionic_character_gaussian_binomial} purely in terms of real-valued trigonometric functions by substituing in Eq. \ref{gaussian_binomial_coefficient}:
	\begin{eqnarray*}
	{2j+1\choose k}_{e^{i\theta}}&=&\frac{\left(1-e^{i(2j+1)\theta}\right)\dots \left(1-e^{i(2j+2-k)\theta}\right)}{\left(1-e^{i\theta}\right)\dots \left(1-e^{ik\theta}\right)}\\
	&=& \frac{\exp(i\theta\sum_{c=1}^k\frac{2j+2-c}{2})}{\exp(i\theta\sum_{c=1}^k\frac{c}{2})}\frac{\prod_{c=1}^k\left(e^{-i(2j+2-c)\theta/2}-e^{i(2j+2-c)\theta/2}\right)}{\prod_{c=1}^k\left(e^{-ic\theta/2}-e^{ic\theta/2}\right)}\\
	&=& \exp(i\theta \left(k(j+1)-\sum_{c=1}^k c\right)) \frac{\prod_{c=1}^k \left(-\frac{1}{2}\sin(\frac{2j+2-c}{2}\theta)\right)}{\prod_{c=1}^k\left(-\frac{1}{2}\sin(\frac{c}{2}\theta)\right)}\\
	&=& e^{-i\left(k(k-1)/2-kj\right)\theta}\prod_{c=1}^k \frac{\sin(\frac{2j+2-c}{2}\theta)}{\sin(\frac{c}{2}\theta)}
	\end{eqnarray*}
	Substituing this into Eq. \ref{multifermionic_character_gaussian_binomial} we obtain
	\begin{equation} \label{multifermionic_character_sine_product}
	\chi^j_k(\theta) = \prod_{c=1}^k\frac{e^{i(2j+2-c)\theta/2}-e^{-i(2j+2-c)\theta/2}}{e^{ic\theta/2}-e^{-ic\theta/2}} = \prod_{c=1}^k \frac{\sin(\frac{2j+2-c}{2}\theta)}{\sin(\frac{c}{2}\theta)}
	\end{equation}
	We see that setting $k=1$ in Eq. \ref{multifermionic_character_sine_product} retrieves Eq. \ref{single_particle_character}.
	
	\subsection{Recursive relation for $\chi^j_k$ from the analogous Pascal identity}
	For Gaussian binomial coefficients, we have two identities as analogues for the binomial Pascal identity (TODO cite this):
	\begin{eqnarray}
	{n\choose k}_q &=& q^k {{n-1}\choose k}_q + {{n-1}\choose {k-1}}_q \label{pascal_identity_1}\\
	{n\choose k}_q &=& {{n-1}\choose k}_q + q^{n-k} {{n-1}\choose {k-1}}_q \label{pascal_identity_2}
	\end{eqnarray}
	Substituing in Eq. \ref{multifermionic_character_gaussian_binomial}, we obtain the following two relations:
	\begin{eqnarray}
	\chi^j_k(\theta) &=& e^{i\frac{k}{2}\theta} \chi^{j-\frac{1}{2}}_k(\theta) + e^{-i\frac{2j+1-k}{2}} \chi^{j-\frac{1}{2}}_{k-1}(\theta)\label{char_pascal_identity_1}\\
	\chi^j_k(\theta) &=& e^{-i\frac{k}{2}\theta} \chi^{j-\frac{1}{2}}_k(\theta) + e^{i\frac{2j+1-k}{2}} \chi^{j-\frac{1}{2}}_{k-1}(\theta) \label{char_pascal_identity_2}
	\end{eqnarray}
	Taking the sum of Eq. \ref{char_pascal_identity_1} and Eq. \ref{char_pascal_identity_2} yields the relation
	\begin{equation} \label{uncooked_pascal_recursion}
	\chi^j_k(\theta) = \frac{1}{2}\left(e^{i\frac{k}{2}\theta}+e^{-i\frac{k}{2}\theta}\right)\chi^{j-\frac{1}{2}}_k(\theta) + \frac{1}{2}\left(e^{-i\frac{2j+1-k}{2}} + e^{i\frac{2j+1-k}{2}}\right) \chi^{j-\frac{1}{2}}_{k-1}(\theta)
	\end{equation}
	Contracting the Dirichlet kernel sum in Eq. \ref{dirichlet_kernel_sum}, we obtain
	\begin{equation} \label{q_plus_q_inverse}
	e^{-i\omega\theta}+e^{i\omega\theta} = \begin{cases}
	\chi^\omega_1(\theta)-\chi^{\omega-1}_1(\theta) & \text{for } \omega \in \left\{1, \frac{3}{2}, 2, \frac{5}{2}\dots\right\}\\
	\chi^\omega_1(\theta) & \text{for } \omega = \frac{1}{2}\\
	2\chi^\omega_1(\theta) & \text{for } \omega = 0
	\end{cases}
	\end{equation}
	Substituing this into Eq. \ref{uncooked_pascal_recursion}, we obtain
	\begin{multline}  \label{pascal_recursion}
	\chi^j_k(\theta) = \frac{1}{2}\left[\left(\chi^{k/2}_1(\theta)-\chi^{k/2-1}_1(\theta)\right)\chi^{j-\frac{1}{2}}_k(\theta)\right.\\
	 \left.+ \left(\chi^{(2j+1-k)/2}_1(\theta)-\chi^{(2j+1-k)/2-1}_1(\theta)\right) \chi^{j-\frac{1}{2}}_{k-1}(\theta)\right]
	\end{multline}
	for $1<k<2j$. Note that if $k=1$, we already have the single-particle state, and if $k=2j$, we can use Eq. \ref{flip_k_by_two_j_plus_one} to immediately retrieve the decomposition.
	
	\section{Angular momentum coupling among fermions}
	In general, an ensemble of $k$ fermions with angular momentum $j$ possesses high rotational symmetry, as such a state has degeneracy $\binom{2j+1}{k}=\frac{(2j+1)!}{k!(2j+1-k)!}$, i.e. the number of subsets of the set of $m$-values with cardinality $k$, which is equal to the number of antisymmetric basis vectors $\ket{b^j_k; \{m_i\}}$. If the Hamiltonian is modified to include a term which couples the angular momenta of the fermions (e.g. orbit\nobreakdash-orbit coupling of electrons on a single atomic orbit), the symmetry is, in general, broken, as the system now possesses a single $m$-value as its quantum number, which is the projection of the \textit{total} angular momentum along the $z$-axis. The new eigenstates of the Hamiltonian will therefore transform as single-particle states under rotation. Therefore, we anticipate there will be a unique decomposition of the following form:
	\begin{equation} \label{general_coupling_decomposition}
	\chi^j_k(\theta)=\sum_{j'}c^j_k(j')\chi^{j'}_{k=1}(\theta)
	\end{equation}
	where $c^j_k(j')$ are the coefficients of single-particle character functions with angular momentum $j'$ into which $\chi^j_k(\theta)$ is decomposed, and which we anticipate to be non-negative integers.
	
	\subsection{Single-particle states form a complete, orthonormal basis}
	For this decomposition to be viable, we require the functions $\chi^j_{k=1}(\theta)$ to form a complete basis of the space spanned by the functions $\chi^j_k(\theta)$. As we prove here, this is true. In particular, we identify the right side of Eq. \ref{single_particle_character} as the Dirichlet kernel $D_j(\theta)$, which in turn is equivalent to the following expression:
	\begin{equation}
	\chi^j_{k=1}(\theta)=U_{2j}\left(\cos{\frac{\theta}{2}}\right)
	\end{equation}
	where $U_{n}(x)$ is the Chebyshev polynomial of the second kind. Since $j=0, \frac{1}{2}, 1, \frac{3}{2}\dots$, we see that the set of single-particle characters is equivalent to the set of Chebyshev polynomials of the second kind. It is known (TODO citation) that Chebyshev polynomials of the second kind form a complete basis for functions on the interval $x\in(-1,1)$. Substituing $x=\cos{\frac{\theta}{2}}$, this turns into the interval $\theta\in (0, 2\pi)$. Even though $\chi^j_k(\theta)$ have in general periodicity with an interval of $4\pi$ (Eq. \ref{character_four_pi_periodic}), they are also all even (Eq. \ref{flip_angle_sign}), and hence they can be fully built up from the quoted interval in a regular manner. As $U_{2j}(\cos{\frac{\theta}{2}})$ must also be even, since $\cos(-\frac{\theta}{2})=\cos{\frac{\theta}{2}}$, this is sufficient to show that $U_{2j}(\cos{\frac{\theta}{2}})$ form a complete basis for the family of character functions of multi-fermionic ensembles.
	
	The Chebyshev polynomials of the second kind also possess an orthogonality relation, which can be readily shown from the orthogonality of sines and cosines and the expression for the Dirichlet kernel, and which is
	\begin{equation} \label{chebyshev_orthogonality}
	\frac{1}{\pi}\int_0^{2\pi} U_{2j}\left(\cos{\frac{\theta}{2}}\right)U_{2j'}\left(\cos{\frac{\theta}{2}}\right) \sin^2{\frac{\theta}{2}}\dd \theta = \delta_{j,j'}
	\end{equation}
	Therefore, if a function $f(\theta)$ is even and $4\pi$-periodic (i.e. $f(x+4\pi)=f(x)$), and hence can be decomposed into the single-particle characters like so:
	\begin{equation}
	f(\theta) = \sum_j c_f(j) \chi^j_1(\theta)
	\end{equation}
	then the decomposition coefficients can be determined using Eq. \ref{chebyshev_orthogonality} thusly:
	\begin{equation}
	c_f(j) = \frac{1}{\pi} \int_0^{2\pi} f(\theta) \chi^j_1(\theta) \sin^2{\frac{\theta}{2}}\dd \theta
	\end{equation}
	Since the integrand is even and $4\pi$-periodic, we can state an equivalent orthogonality relation by extending the intergation domain:
	\begin{equation} \label{orthogonality_four_pi}
	c_f(j) = \frac{1}{2\pi} \int_0^{4\pi} f(\theta) \chi^j_1(\theta) \sin^2{\frac{\theta}{2}}\dd \theta
	\end{equation}
	
	\subsection{Presence of integer and half-integer angular momentum eigenstates in multi-fermionic angular momentum coupling} \label{presence_of_half_integer}
	Let us consider the behaviour of Eq. \ref{general_coupling_decomposition} under the rotation by $2\pi+\theta$:
	\begin{equation} 
	\chi^j_k(2\pi+\theta)=\sum_{j'}c^j_k(j')\chi^{j'}_{k=1}(2\pi+\theta)
	\end{equation}
	Using Eq. \ref{shift_by_two_pi}, we obtain
	\begin{equation} 
	(-1)^{2jk}\chi^j_k(\theta)=\sum_{j'}(-1)^{2j'}c^j_k(j')\chi^{j'}_{k=1}(\theta)
	\end{equation}
	Hence the following decomposition is also true:
	\begin{equation}  \label{shifted_decomposition}
	\chi^j_k(\theta)=\sum_{j'}(-1)^{2(j'-jk)}c^j_k(j')\chi^{j'}_{k=1}(\theta)
	\end{equation}
	However, since the basis formed by $\chi^j_{k=1}(\theta)$ is orthonormal, a decomposition is unique. Hence Eq. \ref{shifted_decomposition} must be equivalent to Eq. \ref{general_coupling_decomposition}:
	\begin{equation}
	(-1)^{2(j'-jk)}c^j_k(j') = c^j_k(j')
	\end{equation}
	If $2(j'-jk)$ is even, this trivially holds. If $2(j'-jk)$ is odd, this only holds if $c^j_k(j')=0$. Hence the decomposition coefficients may be non-zero only in the following cases:
	\begin{enumerate}
	\item If $j$ is half-integer \textbf{and} $k$ is odd, $j'$ must be half-integer, too.
	\item Otherwise, $j'$ must be an integer.
	\end{enumerate}
	For the values of $j'$ not covered in these two cases, the coefficients $c^j_k(j')$ are zero.
	
	Hence, if an ensemble of $k$ fermions occupying a specific $j$-eigenstate undergoes angular momentum coupling, then:
	\begin{enumerate}
	\item if $jk$ is a half-integer, the state decomposes into single-particle states with \textit{only} half-integer angular momenta;
	\item if $jk$ is an integer, the state decomposes into single-particle states with \textit{only} integer angular momenta.
	\end{enumerate}
	
	\subsection{The decomposition of multi-fermionic states under $j$-coupling as coefficients of Gaussian binomial coefficient expansion}
	The Gaussian binomial coefficient (as defined in Eq. \ref{gaussian_binomial_coefficient}) is a polynomial in $q$, which can be shown (TODO cite) to have positive, integer coefficients. It can also be shown (TODO cite) that the degree of this polynomial is $k(n-k)$. Hence, there exists an expansion:
	\begin{equation}
	{n\choose k}_q=\sum_{m=0}^{k(n-k)} \upsilon^n_k(m) q^m\qq{where} \upsilon^n_k(m)\in\mathbb{N}^+
	\end{equation}
	Substituting this expansion into Eq. \ref{multifermionic_character_gaussian_binomial} yields
	\begin{equation} \label{naive_sum_decomposition}
	\chi^j_k(\theta) = \exp(-i\frac{k(2j+1-k)}{2}\theta)\sum_{m=0}^{k(2j+1-k)} \upsilon^n_k(m) e^{im\theta}
	\end{equation}
	It can be shown (TODO cite) that $\upsilon^n_k(m)$ is equal to the number of $k$-element subsets of the set $\gamma_n=\{1, 2\dots n\}$ whose sum is $k(k+1)/2+m$. Then, consider a $k$-element subset of $\gamma_n$:
	$$\{a_1, a_2\dots a_k\}\qq{where} \sum_{i=1}^k a_i = \frac{k(k+1)}{2} + m \qq{for some} m$$
	Let us construct a new $k$-element subset of $\gamma_n$ by taking $b_i=n+1-a_i$:
	$$\{n+1-a_1, n+1-a_2\dots n+1-a_k\}\qq{where} \sum_{i=1}^k a_i = \frac{k(k+1)}{2} + m$$
	Then the sum of this new subset is equal to
	\begin{equation}
	\sum_{i=1}^k b_i = \sum_{i=1}^k (n+1) - \sum_{i=1}^k a_i = \frac{k(k+1)}{2}+(k(n-k)-m)
	\end{equation}
	Hence for values of $m$ for which $m\neq k(n-k)-m$, there exists a bijection between the $k$-element subsets which sum up to $k(k+1)/2+m$ and those which sum up to $k(k+1)/2+(k(n-k)-m)$. The exception occurs when
	\begin{eqnarray}
	m &=& k(n-k)-m\\
	m &=& \frac{k(n-k)}{2}
	\end{eqnarray}
	We shall refer to this value as singular $m\equiv m_s = k(n-k)/2$
	Hence we have
	\begin{equation} \label{reflective_q_gaussian_coefs}
	\upsilon^n_k(m) = \upsilon^n_k(k(n-k)-m) \qq{for} m\neq \frac{k(n-k)}{2}
	\end{equation}
	It can also be shown (TODO cite) that $\upsilon^n_k(m)$ are unimodal in $m$. From unimodality and the property in Eq. \ref{reflective_q_gaussian_coefs}, we see that if the value $m=m_s$ is present in the terms of the sum in Eq. \ref{naive_sum_decomposition}, it must be the mode of $\upsilon^n_k(m)$ (i.e. the largest element in the distribution of $e^{im\theta}$). If it is not present, the two elements closest to the value, which are equal, will be the two largest coefficients.
	
	Now, let us consider the case $n=2j+1$, where $j$ can be an integer or a half-integer, and $k$ is always an integer. We shall inspect the terms of the sum in Eq. \ref{naive_sum_decomposition} for the following two cases:
	\begin{enumerate}
	\item If $k(2j+1-k)$ is even, there will be a term present for which ${m=m_s}$, and Eq. \ref{naive_sum_decomposition} can be written as
	\begin{equation} \label{q_gauss_integer_j_decomposition}
	\chi^j_k(\theta) = \upsilon^{2j+1}_k\left(m_s\right) + \sum_{m=1}^{m_s} \upsilon^{2j+1}_k\left(m_s+m\right)\left(e^{im\theta}+e^{-im\theta}\right)
	\end{equation}
	\item If $k(2j+1-k)$ is odd, $m_s$ is half-integer, no term will be present for $m=m_s$, and Eq. \ref{naive_sum_decomposition} can be written as
	\begin{equation} \label{q_gauss_half_integer_j_decomposition}
	\chi^j_k(\theta) = \sum_{m=\frac{1}{2}}^{m_s} \upsilon^{2j+1}_k\left(m_s+m\right)\left(e^{im\theta}+e^{-im\theta}\right)
	\end{equation}
	\end{enumerate}
	In the previous two sums, we have used Eq. \ref{reflective_q_gaussian_coefs} in the form
	\begin{equation}
	\upsilon^{n}_k(m_s+m)=\upsilon^{n}_k(m_s-m)
	\end{equation}
	Now: suppose $k(2j+1-k)$ is even. We expand this into $2kj-k(k+1)$ is even. However, $k(k-1)$ is even for any integer $k$. Hence the statement is equivalent to $2jk$ being even, or $jk$ being integer. From this we see that the decompositions in Eq. \ref{q_gauss_integer_j_decomposition}, resp. Eq. \ref{q_gauss_half_integer_j_decomposition} coincides with $\chi^j_k(\theta)$ decomposing into purely integer, respectively purely half-integer single-particle states, as per Sec. \ref{presence_of_half_integer}.
	
	Now, we substitute Eq. \ref{q_plus_q_inverse} into Eq. \ref{q_gauss_integer_j_decomposition} and Eq. \ref{q_gauss_half_integer_j_decomposition}. This immediately retrieves the coefficients from Eq. \ref{general_coupling_decomposition} in the angular-momentum-coupling decomposition:
	\begin{enumerate}
	\item If $jk$ is integer:
	\begin{equation}
	\chi^j_k(\theta) = \upsilon^{2j+1}_k\left(m_s\right)\chi^{j=0}_1(\theta) + \sum_{m=1}^{m_s} \upsilon^{2j+1}_k\left(m_s+m\right)\left(\chi^m_1(\theta)-\chi^{m-1}_1(\theta)\right)
	\end{equation}
	Collecting the terms $\chi^{j'}_1(\theta)$ for a specific $j'$:
	\begin{equation}
	c^j_k(j') = \upsilon^{2j+1}_k\left(m_s+j'\right)-\upsilon^{2j+1}_k\left(m_s+j'+1\right)\qq{where}j'=0,1,2\dots
	\end{equation}
	\item If $jk$ is half-integer:
	\begin{equation}
	\chi^j_k(\theta) = \upsilon^{2j+1}_k\left(m_s+\frac{1}{2}\right)\chi^{j=1/2}_1(\theta) + \sum_{m=\frac{3}{2}}^{m_s} \upsilon^{2j+1}_k\left(m_s+m\right)\left(\chi^m_1(\theta)-\chi^{m-1}_1(\theta)\right)
	\end{equation}
	Collecting the terms $\chi^{j'}_1(\theta)$ for a specific $j'$:
	\begin{equation}
	c^j_k(j') = \upsilon^{2j+1}_k\left(m_s+j'\right)-\upsilon^{2j+1}_k\left(m_s+j'+1\right)\qq{where}j'=\frac{1}{2},\frac{3}{2},\frac{5}{2}\dots
	\end{equation}
	\end{enumerate}
	We see that the coefficients $c^j_k(j')$ obey the same relation for both cases, and the only thing that differs is the subset of $\frac{1}{2}\mathbb{N}$ on which the coefficients are trivially zero. Hence we can write down the most general decomposition of a multi-fermionic state into single-particle states under angular momentum coupling:
	\begin{multline} \label{decomposition_via_gaussian_binomials}
	c^j_k(j') = \upsilon^{2j+1}_k\left(\frac{k(2j+1-k)}{2}+j'\right)-\upsilon^{2j+1}_k\left(\frac{k(2j+1-k)}{2}+j'+1\right)\\
	\qq{where} \upsilon^n_k\left(x\right)\qq{is the coefficient of} q^x \qq{in} {n\choose k}_q
	\end{multline}
	Thus the decomposition of multi-fermionic states into single-particle states is fully described by the coefficients of expanded Gaussian binomials.
	
	\subsection{Recursive relation for the decomposition coefficients from the analogous Pascal identity}
	We shall utilise Eq. \ref{pascal_recursion} to state a recursive relation for $c^j_k(j')$ in terms of $c^{j-1/2}_{k-1}(j')$ and $c^{j-1/2}_{k}(j')$.
	
	First, let us state the usual relation that governs the angular momentum coupling between two distinguishable particles:
	
	\begin{equation} \label{distinguishable_particle_coupling_decomposition}
	\chi^a_1(\theta)\chi^b_1(\theta)=\sum_{j=|a-b|}^{a+b} \chi^j_1(\theta)
	\end{equation}
	This relation can be easily proven from Eq. \ref{dirichlet_kernel_sum}, where the product of the two sums can be treated as a convolution of their Fourier series via the convolution theorem.
	
	Now, consider the following product:
	\begin{align}
	\left(\chi^a_1(\theta)-\chi^{a-1}_1(\theta)\right)\chi^b_1(\theta)&\qq{where}a\geq 1\nonumber\\
	&= \sum_{j=|a-b|}^{a+b}\chi^j_1(\theta)-\sum_{j=|a-b-1|}^{a+b-1}\chi^j_1(\theta)
	\end{align}
	
	\subsection{The multi-fermionic rotation character as a function of single-particle rotation characters}
	The expression in Eq. \ref{ensemble_character_recursive_relation} expresses $\chi^j_k(\theta)$ as a function of all $\chi^j_{k'}(\theta)$ for $k'<k$. To eliminate the complexity of this recursive relationship, let us first change the summation index $a\rightarrow k-a$ and then recognise $\Tr d^j(\phi)=\chi^j_1(\phi)$:
	\begin{equation}
	\chi^j_k(\theta)=-\frac{1}{k}\sum_{a=0}^{k-1} (-1)^{k-a}\chi^j_1((k-a)\theta)\chi^j_{a}(\theta)
	\end{equation}
	Taking the term $\chi^j_{a}(\theta)$ and expressing it as a sum again, we obtain a nested sum expression:
	\begin{equation} \label{nested_sum_a}
	\chi^j_k(\theta)=-\frac{1}{k}\sum_{a_1=0}^{k-1} (-1)^{k-a_1}\chi^j_1((k-a_1)\theta)\frac{-1}{a_1}\sum_{a_2=0}^{a_1-1} (-1)^{a_1-a_2}\chi^j_1((a_1-a_2)\theta)\frac{-1}{a_2}\sum_{a_3=0}^{a_2-1}\dots
	\end{equation}
	where the nested sum terminates at $a_x=0$ for some $x$ dependent on the shape of the array $a_1, a_2\dots$, at which point we directly evaluate the term $\chi^j_{a_x=0}(\theta)=1$.
	
	To simplify this nested sum, let us choose a new set of indices:
	\begin{eqnarray}
	b_i &=& a_{i-1} - a_i\\
	b_1 &=& k - a_1
	\end{eqnarray}
	Let us define the partial sum of the array $b_i$ like so:
	\begin{eqnarray}
	B_i &=& \sum_{x=1}^i b_x\\
	B_0 &=& 0
	\end{eqnarray}
	This allows us to write the inverse of our index transformation:
	\begin{equation}
	a_i = k - B_i
	\end{equation}
	If the sum over $a_i$ ranges from $0$ to $a_{i-1}-1$, this corresponds to summing over $b_i$ ranging from $1$ to $a_{i-1}=k-B_{i-1}$, where for $b_1$ the upper limit is $a_0\equiv k$.
	
	Then, Eq. \ref{nested_sum_a} becomes
	\begin{equation}
	\chi^j_k(\theta) = \frac{-1}{k-B_0}\sum_{b_1=1}^k (-1)^{b_1} \chi^j_1(b_1\theta)\frac{-1}{k-B_1}\sum_{b_2=1}^{k-B_1}\dots
	\end{equation}
	where the sum now terminates at the point $B_x = k$, where we directly evaluate $\chi^j_{k-B_x}(\theta)=\chi^j_{0}(\theta)=1$. Hence, we can write the expression for $\chi^j_k(\theta)$ purely as a function of $\chi^j_1(x)$ in the following way:
	\begin{equation}
	\chi^j_k(\theta) = \sum_{(b_1, b_2\dots b_n)}\prod_{i=1}^n (-1)^{1+b_i}\frac{\chi^j_1(b_i\theta)}{k-B_{i-1}}
	\end{equation}
	We can simplify this further by taking all the factors of $-1$ out of the product and recognising the sum over all $b_i$ as $B_n=k$:
	\begin{equation} \label{partitioning_arrays_expression}
	\chi^j_k(\theta) = (-1)^k\sum_{(b_1, b_2\dots b_n)}(-1)^n\prod_{i=1}^n \frac{\chi^j_1(b_i\theta)}{k-B_{i-1}}
	\end{equation}
	where the sum is over all ordered lists $(b_1, b_2\dots b_n)$ which are integer compositions of $k$, i.e. $b_i\geq 1$ and $\sum_ib_i = k$.
	
	\subsection{Single-particle decomposition coefficients of multi-fermionic states as sums}
	Inserting Eq. \ref{dirichlet_kernel_sum} into Eq. \ref{partitioning_arrays_expression}, we obtain:
	\begin{equation}
	\chi^j_k(\theta) = (-1)^k\sum_{(b_1, b_2\dots b_n)}(-1)^n\left(\prod_{i=1}^n \frac{1}{k-B_{i-1}}\right)\prod_{i=1}^n \sum_{c=-j}^j e^{icb_i\theta}
	\end{equation}
	Let us denote the coefficients in the expansion of the Dirichlet kernel product by a special function:
	\begin{equation} \label{omega_distribution_decomposition}
	\prod_{i=1}^n \sum_{c=-j}^j e^{icb_i\theta}=\sum_{\omega=0,\pm\frac{1}{2},\pm 1,\pm\frac{3}{2}\dots}\Omega^j_{(b_i)}(\omega)e^{i\omega\theta}
	\end{equation}
	Let us also take notice of the following identity:
	\begin{equation}
	\sin^2(\frac{\theta}{2})=\left(\frac{i}{2}\left(e^{i\frac{\theta}{2}}-e^{-i\frac{\theta}{2}}\right)\right)^2=\frac{1}{2}\left(1-\frac{1}{2}\left(e^{i\theta}+e^{-i\theta}\right)\right)
	\end{equation}
	
	Now, we use Eq. \ref{orthogonality_four_pi} with Eq. \ref{dirichlet_kernel_sum} substituted in for the single-particle character function to obtain the decomposition coefficients of a general multi-fermionic rotation character:
	\begin{eqnarray*}
	c^j_k(j')&=&\frac{(-1)^k}{4\pi}\sum_{(b_1, b_2\dots b_n)}(-1)^n\left(\prod_{i=1}^n \frac{1}{k-B_{i-1}}\right)\sum_{\omega}\Omega^j_{(b_i)}(\omega)\sum_{c'=-j}^j\int_0^{4\pi}\left(\right.\\
	&&\left. e^{i(c'+\omega)\theta}-\frac{1}{2}\left(e^{i(c'+\omega + 1)\theta}+e^{i(c'+\omega -1)\theta}\right) \right)\dd\theta
	\end{eqnarray*}
	where
	\begin{equation}
	c^j_k(\theta)=\sum_{j'=0,\frac{1}{2},1\dots}c^j_k(j')c^{j'}_1(\theta)
	\end{equation}
	Now, for a general integral of $\exp(ia\theta)$ for half-integer $a$, we obtain:
	\begin{equation}
	\int_0^{4\pi}e^{ia\theta}\dd\theta=2\int_0^{2\pi}e^{i2a\phi}\dd\phi=4\pi\delta_{2a}=4\pi\delta_a
	\end{equation}
	where $\delta_a$ is a short-hand for the Kronecker delta $\delta_{a,0}$. Here we used a substitution $\phi=\frac{\delta}{2}$ and the fact that $2a$ is integer. This is applicable to all three terms inside the integral expression for $c^j_k(j')$, which allows us to write
	\begin{eqnarray*}
	c^j_k(j')&=&(-1)^k\sum_{(b_1, b_2\dots b_n)}(-1)^n\left(\prod_{i=1}^n \frac{1}{k-B_{i-1}}\right)\sum_{\omega}\Omega^j_{(b_i)}(\omega)\sum_{c'=-j}^j\left(\right.\\
	&&\left. \delta_{c'+\omega}-\frac{1}{2}\left(\delta_{c'+\omega + 1}+\delta_{c'+\omega -1}\right) \right)\dd\theta
	\end{eqnarray*}
	As we can see, for a Kronecker delta term of the form $\delta_{c'+\omega}$, this term ends up contributing to the sum if and only if $j'$ is equal to $|\omega|$ or greater by an integer interval (so that the sum over $c'$ contains the value $-\omega$). Hence we see that for a term with a specific value $\omega$, the contribution to terms with $j'=|\omega|+1$ or greater will be zero, as all Kronecker deltas evaluate to $1$ exactly once in the sum over $c'$, leaving their total to be $1-\frac{1}{2}(1+1)=0$. Conversely, the contribution to the terms $j'=|\omega|-2$ or smaller will also be zero, as all of the Kronecker deltas will have zero contribution. This allows us to write the contributions to a term with a specific value of $j'$ in terms of the numbers of terms with specific values of $\omega$, which are given by the distribution $\Omega^j_{(b_i)}(\omega)$:
	\begin{itemize}
	\item For $j'=0$, the terms with $\omega=0$ contribute with a weight of $1$, and terms with $|\omega|= 1$ contribute with a weight of $-\frac{1}{2}$ (since only one of the two negative Kronecker deltas is contained within the sum over $c'$).
	\item For $j'=\frac{1}{2}$, the terms with $|\omega|=\frac{1}{2}$ contribute with a weight of $\frac{1}{2}$ (since it is the positive and one of the negative Kronecker deltas which are contained in the sum over $c'$), and the terms with $|\omega|=\frac{3}{2}$ contribute with a weight of $-\frac{1}{2}$, because only one of the two negative Kronecker deltas is contained in the sum.
	\item For $j'=1$, we have a contribution from $|\omega|=1$ and $|\omega|=2$ with weights $\frac{1}{2}$, $-\frac{1}{2}$, respectively, as per the next bullet point. Note that the contribution from $\omega = 0$ is zero, because even though the term $\omega-1$ has a positive absolute value, the absolute value is equal to one and as such is contained in the sum over $c'$, and thus every Kronecker delta term will have one contribution in the sum, adding up to zero.
	\item For larger $j'$, there will be a contribution from terms with $|\omega|=j'$ with a weight of $\frac{1}{2}$, since one of the negative Kronecker deltas is \textit{not} included in the sum over $c'$; there will also be a contribution from terms with $|\omega|=j'+1$ with a weight of $-\frac{1}{2}$, since one of the negative Kronecker deltas \textit{will} be included in the sum over $c'$ (and the positive Kronecker delta will not be included).
	\end{itemize}
	Now, let us consider the expansion of Eq. \ref{omega_distribution_decomposition}:
	\begin{equation}
	\prod_{i=1}^n \sum_{c=-j}^j e^{icb_i\theta}=\sum_{c_1=-j}^j\dots\sum_{c_n=-j}^j \exp(i\left(\sum_{i=1}^n b_ic_i\right)\theta)
	\end{equation}
	For each term with $\omega = \left(\sum_{i=1}^n b_ic_i\right)\neq 0$, we can find a unique conjugate term by taking $c_i\to-c_i$. This also takes $\omega\to-\omega$. Hence, there is a bijection between the terms with $\omega$ and the terms with $-\omega$, leaving us with the conclusion that
	\begin{equation}
	\Omega^j_{(b_i)}(-\omega)=\Omega^j_{(b_i)}(\omega)
	\end{equation}
	Applying this to the contributions to each $j'$ term listed above, we can write them all concisely like so:
	\begin{equation}
	c^j_k(j')=(-1)^k\sum_{(b_1, b_2\dots b_n)}(-1)^n\left(\prod_{i=1}^n \frac{1}{k-B_{i-1}}\right)\left(\Omega^j_{(b_i)}(j')-\Omega^j_{(b_i)}(j'+1)\right)
	\end{equation}
	This is true for all half-integer, non-negative values of $j'$.
	
	\begin{thebibliography}{10}



\end{thebibliography}	
	
\end{document}