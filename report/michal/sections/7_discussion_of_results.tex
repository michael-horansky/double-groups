\section{Discussion of results}

%\subsection{Symmetry suppression in pyramidal quantum dots}
%The suppression theory in \ref{sec:suppression} is discussed in the context of the quantum dots grown in the mode described in \ref{sec:growth}. The limiting symmetry of the nanocrystallic structure is shown. The effects of strain and other causes of perturbation are discussed. The $C_{3v}$ probing for the $X_{\bar{1},1}$ transition is discussed and explanations are offered.
no agreement, too many lines for Td

\subsection{Agreement with $C_{6v}$ is a coincidence}
its a mathematical coincidence. C6v cannot be the group because fcc lattice. no other group agrees with experiment.

%\subsection{Other hypothetical causes of symmetry elevation}
%This section serves to enumerate another effects we have identified which might cause symmetry breaking or potential elevation in QDs. These include band-mixing effects and electron gas models. Since they are not mathematically developed, my goal is to identify specifically the aspects of these hypotheses that might for a basis for a change of symmetry. I also discuss why the interplay between these effects can be neglected, as that would constitute second-order effects (and I justify this claim by qualitative arguments).

\subsection{Symmetry outside of the $\Gamma$ point}
In the symmetry suppression model, we have assumed all excited fermions to have zero crystal momentum, which is necessary to use the "charged particles in a box" dynamics for the envelope functions. However, this is only an approximation based on the nature of the direct band-gap of InGaAs. For an excited fermion outside of the $\Gamma$ point, the non-zero crystal momentum vector $\vec{k}$ breaks the wavefunction symmetry, as described by Dresselhaus in \cite[Ch. 13]{dresselhaus}. In an interplay with the bulk symmetry, the true symmetry of the Hamiltonian $\hat{H}\left(\vec{k}\right)$ depends on the crystal momentum.

Since this effect breaks symmetry and does not elevate it, it is sufficient to test all subgroups of the Hamiltonian symmetry at $\Gamma$ to see if the predictions for any of them match the experimental data. Obtaining such a group and finding the values of $\vec{k}$ in the Brillouin zone it corresponds to could hint at a possible effect which forces the excitons to have a non-zero crystal momentum.

acctually the envelope may just transform according to smth

\subsection{mottness? spontaneous symmetries? what?}
