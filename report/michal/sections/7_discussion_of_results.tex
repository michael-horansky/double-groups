\section{Discussion of results}

%\subsection{Symmetry suppression in pyramidal quantum dots}
%The suppression theory in \ref{sec:suppression} is discussed in the context of the quantum dots grown in the mode described in \ref{sec:growth}. The limiting symmetry of the nanocrystallic structure is shown. The effects of strain and other causes of perturbation are discussed. The $C_{3v}$ probing for the $X_{\bar{1},1}$ transition is discussed and explanations are offered.

\subsection{Agreement with the bulk symmetry $T_d$}
Observing which excitons seem to elevate to $T_d$ generates predictions within our theory about the localisation of their wavefunctions within the QD bulk. The excitons which were observed to undergo symmetry elevation to $T_d$ in the MCJP dataset are $X_{\bar{1}0}$ (for one sub-population), $X^-_{\bar{1}0}$ (not in the Karlsson data), $2X^+_{\bar{2}1}$ (with one peak being ambiguously identified to the exciton), $X^+_{\bar{1}1}$ (for one sub-population),  $X^+_{1\bar{1}}$ (with the relative intensity of the polarisations not explicable with symmetry arguments), and $2X_{\bar{2}0}$ (assuming the $6$ lines are unresolvable due to tight spacing). Conversely, the excitons $X_{0\bar{1}}$ (for one sub-population), $X^+_{\bar{1}1}$ (for one sub-population), and $X^+_{1\bar{1}}$ (only in the Karlsson data) seem to explicitly disconfirm $T_d$ symmetry and probe for the structure symmetry. It is important to note that the sub-population divisions for the neutral excitons and for the positive excitons are different.

It seems that all excitons with only heavy holes undergo symmetry elevation to the bulk, in agreement with Sec. \ref{sec:localisation}. The ambiguity for mixed-hole excitons suggests that light holes indeed may probe the structure symmetry due to their delocalisation. Neither MCJP not Karlsson \textit{et al.} were able to resolve the light hole-dominant excitons for which the predictions of $C_{3v}$ and $T_d$ differ ($X^+_{0\bar{2}}$, $2X_{0\bar{2}}$, and positive biexcitons with $2$ or $3$ light holes). Resolving these transitions is important material for further study, since if they behave according to structure symmetry, it would be evidence that the symmetry suppression   model correctly identifies the mechanism for symmetry elevation.

One important phenomenon disregarded in our treatment is the effect of structure interfaces, which affect the Hamiltonian symmetry group \cite{interfaces}. Whilst a full treatment of the symmetry suppression model has to address this to prove elevation to $T_d$ is still possible, this could also validate our data e.g. by providing the polarisation bias to $X^+_{1\bar{1}}$, explaining the relative line intensities.

\subsection{Insufficient treatment of line number increase} \label{sec:failed_degeneracy}
The model of symmetry suppression postulates that excitons with wavefunctions localised in the QD bulk enjoy approximate elevation to the bulk (lattice symmetry). Within this model, lines which are light in the lower symmetry but dark in the elevated symmetry will be heavily suppressed. However, the opposite effect (where the number of lines increases with symmetry elevation), which occurs for certain excitons when elevating to $T_d$, is non-trivial. Specifically, for every exciton with two holes, this is mathematically caused by the fact that in $T_d$, the heavy hole and light hole bands are degenerate at $\Gamma$ and as such a filled energy level on one of the valence bands becomes half-empty, becoming $6$ orthogonal states distributed in $3$ energy levels of varying degeneracies (Table \ref{tab:multihole_states}). Also, one-hole and three-hole states are now quadruply, not doubly degenerate, often leading to more different energy levels under $j$-coupling. The problem is that in the symmetry suppression model, the heavy hole and light hole band separation at $\Gamma$ does get suppressed, but it cannot vanish. The mathematical treatment of this \textit{approximately-degenerate} band structure is not yet developed. However, there are many low-order effects which were disregarded in the simplistic, first-order theory we used, which could potentially allow the different-character holes couple together in a way that salvages the degenerate behaviour within our model. This is material for further study.

\subsection{Symmetry outside of the $\Gamma$ point}
In the symmetry suppression model, we have assumed all excited fermions to have zero crystal momentum, which is necessary to use the "charged particles in a box" dynamics for the envelope functions. However, this is only an approximation based on the nature of the direct band-gap of InGaAs. For an excited fermion outside of the $\Gamma$ point in a large crystal, the non-zero crystal momentum vector $\vec{k}$ breaks the wavefunction symmetry, as described by Dresselhaus in \cite[Ch. 13]{dresselhaus}. In an interplay with the bulk symmetry, the true symmetry of the Hamiltonian $\hat{H}\left(\vec{k}\right)$ depends on the crystal momentum. 

In our nanocrystal, the effects of the crystal size alter the band structure, and a full description of the wavefunction is not possible with the crystal momentum, but rather with the shape of the envelope function, which is poorly understood. Exciting the envelope function from the ground state (i.e. leaving the $\Gamma$ point) may break some symmetries, but in general, it also changes the transformation properties of the wavefunction, since $\chi^{\left(\Upsilon\left(\vec{r}\right)\right)}\left(\hat{R}\right)$ in eq. \ref{eq:character_breakdown} is no longer always unity. This could potentially explain the division of QDs into two sub-populations for some exciton complexes (and the discrepancy with the Karlsson data), where the distribution of envelope function energies would become a non-trivial statistical phenomenon. However, the shapes of the envelope functions are not yet well understood, as they are highly sensitive to high-order or mechanical effects, such as strain-induced piezoelectricity (Fig. \ref{fig:lens_envelopes}).

Of course, since the extra degree of freedom in transformation properties could alter any prediction for photoluminiscence spectrum, it may also serve as an alternative explanation to symmetry elevation.\\

\makebox[0pt][l]{%
\begin{minipage}{\textwidth}
\centering
    \includegraphics[width=\textwidth]{figures/pyramid_envelopes}
 \captionof{figure}{Envelope functions of ground- and excited-state electrons and holes in a $C_{4v}$ pyramid structure for first-order theory and including strain-induced piezoelectric effect. Figure taken from \cite[Fig. 6]{bulk_limiting}}
 \label{fig:lens_envelopes}
\end{minipage}
}