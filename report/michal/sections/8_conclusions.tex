\section{Conclusions} \label{sec:conclusions}

Based on the arguments in \cite{bulk_limiting} and our analysis of the zincblende lattice of InGaAs quantum dots, we claim that neither of the symmetry groups proposed by Karlsson \textit{et al.} ($D_{3h}, C_{6v}$) can be the symmetry group of the Hamiltonian under symmetry elevation. In our model, where the extra symmetries are provided by the atomic lattice in the nanocrystal bulk, we have outlined a general condition for a possible elevated-symmetry group given by the true symmetry and the bulk symmetry (Eq.~(\ref{eq:group_chain})). In general, the true symmetry group $G_s$ may be lower than the structure symmetry due to mismatching orientation of rotations for structure symmetries and bulk symmetries or \textit{a priori} symmetry breaking. The elevated symmetry must still be a subgroup of, or equal to the bulk symmetry.

Applying this method, we have identified $T_d$ as the only candidate group which $C_{3v}$ can elevate to. Comparing the predictions for emission spectra arising from this group to the to-be published MCJP data, we find partial agreement, suggesting certain excitons undergoing elevation to $T_d$ for at least a certain fraction of the grown QDs.

The true symmetry of the Hamiltonian may be a different subgroup of $T_d$ if prior symmetry breaking occurs in the system, e.g. due to interface effects, piezo-electric or other strain-induced effects, excited-state envelope functions, or other untreated physical phenomenons. Testing the predictions given by subgroups of $T_d$ which do not contain $C_{3v}$ is material for further study, as finding agreement with such a group can also provide clues about the nature of symmetry breaking in the system.

We have constructed a theoretical model of symmetry suppression in Sec.~\ref{sec:symmetry_suppression} as a proposed mechanism for elevating towards bulk symmetry. The model suffices to explain the approximate darkening (suppression) of emission lines as identified by Karlsson \textit{et al.}, but fails to justify the raised degeneracy of valence bands at $\Gamma$ which is required to reach $T_d$ symmetry (Sec.~\ref{sec:failed_degeneracy}). A more robust treatment, possibly incorporating higher-order effects and band-mixing effects is needed to justify this corollary of symmetry elevation to $T_d$.

Within the model of symmetry suppression, excitons are more likely to elevate to bulk symmetry if more of their holes are heavy-like. We conclude that this is untrue for positively charged excitons, but holds for all other exciton and biexciton complexes, hinting at a mechanism affecting wavefunction localisations which is untreated in our simplistic model in Sec.~\ref{sec:localisation}. The exciton transitions $X^+_{0\bar{2}}$, $2X_{0\bar{2}}, 2X^+_{\bar{1}2}, 2X^+_{1\bar{2}}, 2X^+_{0\bar{3}}$ were not yet experimentally resolved, and if future studies demonstrate these transitions to behave according to the structure symmetry, it would both provide evidence that symmetry suppression is the mechanism for symmetry elevation and predict within the model that the associated exciton complexes are not localised in the centre of the quantum dot.

As demonstrated by the discrepancies between the MCJP and Karlsson datasets, as well as the different behaviours of distinct QDs within each dataset, the features of the photoluminiscence spectra of these structures cannot be reliably predicted only by arguments based on the structure and bulk symmetry properties. While a good stepping stone, a fuller understanding of both the symmetry of the true Hamiltonian and other effects which may affect the photoluminiscence spectrum is needed.
