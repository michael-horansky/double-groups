\documentclass[12pt]{article}
\usepackage{stdheadstart}
\usepackage{xargs}
\usepackage{physics}
\usepackage{amsmath,amssymb}



\begin{document}

	\title{Symmetry suppression of allowed transitions between energy eigenstates of a Hamiltonian perturbed from a high symmetry.}	
	\maketitle
	
	\abstract{It has been observed in quantum dots with $C_{3v}$ symmetry that certain electronic transitions allowed by the dipole approximation selection rule are unresolvable in the measured photoluminiscence (PL) spectrum. We provide a theoretical model that predicts these spectral lines to be severely weakened if the Hamiltonian is considered to be perturbed from one with a higher spatial symmetry. An expression for the suppressing factor of the spectral line intensity is given in 1st order perturbation theory.}
	
	\section{Introduction}
	Karlsson et al \cite{karlsson} have studied the PL spectrum of InGaAs $C_{3v}$-symmetric quantum dots (QDs). Their model assumes that the QD-Hamiltonian eigenstates which interact with electromagnetic waves on the studied interval of the spectrum are given by considering the symmetry breaking of exciton complex fine structure, which splits total angular momentum $J$ eigenstates into energy levels which transform as double group irreps. By using rudimentary group theory, the energy levels, their degeneracies, and their allowed transitions can be calculated for all excitonic complexes, as was done by this particular QD. However, it was found that the number of spectral lines resolvable in a spectral feature associated with a specific excitonic transition is smaller than the predicted number for several exciton complex transitions, namely $X_{10}\to vac., 2X_{20}\to X_{10},2X_{11}\to X_{10}, 2X_{11}\to X_{01}, 2X^+_{21}\to X^+_{11}$. Since the number of allowed transitions lowers if the Hamiltonian is altered to commute with more rotations (e.g. to $D_{3h}$ or $C_{6v}$ point group symmetry), the effect was dubbed "symmetry elevation". However, no explanation for why symmetry elevation occurs was found.
	
	What the previous work on symmetry elevation in QDs did not consider is the nanocrystallic lattice that constitutes the QDs inside the potential well. The InGaAs QDs are formed by a lattice which does possess a 6-fold rotation along the $z$-axis. Even though the full Hamiltonian still only possesses $C_{3v}$ symmetry, since it is ultimately dominated by the lower-symmetry potential well boundary shape, it can be modelled as a high-symmetry Hamiltonian which underwent symmetry breaking with a small perturbation and enjoys certain approximate symmetries, which still turn out to have a considerate effect in 1st order expansion. To obtain a theoretical description of the effects of these approximate symmetries, we first consider 1st order perturbation theory and describe a surjective mapping of basis vectors of irreps of the larger point group onto the basis vectors of irreps of its point subgroup, which allows us to determine the mapping of high-symmetry Hamiltonian eigenstates onto the low-symmetry Hamiltonian eigenstates under symmetry breaking, and then we emply Fermi's golden rule to quantify the suppression of certain spectral lines. The manifestation of de-intensifying of specific spectral lines due to approximate symmetries will be named "symmetry suppression".
	
	\section{Theoretical model of symmetry suppression}
	We shall first outline a general approach to describing a non-specific minimally symmetry-broken Hamiltonian, and then argue for why the assumptions made are justified in the case of InGaAs QDs. 
	\subsection{Symmetry breaking as basis vector surjection}
	
	picture here
	\subsection{Domain alteration and perturbation theory}
	Suppose now we have a system described by a Hamiltonian $\hat{H}$ equipped with a group of symmetries $G$. Let us construct a Hamiltonian $\hat{H}_+$ which is equipped with a group of symmetries $G_+$ such that $G<G_+$ and $\hat{H}$ can be modelled as a perturbation of $\hat{H}_+$ like so:
	$$\hat{H}=\hat{H}_+ + \lambda \hat{H}',\qq{where}\lambda\ll 1, \mel{E_i;n}{\hat{H}'}{E_i;n}\sim E_i$$
	Let us denote the eigenstates of the Hamiltonians $\hat{H},\hat{H}_+$ as $\ket{E_i;n}$ and $\ket{E^+_{i'};n'}$ respectively, where $n, n'$ are the degeneracy labels. Disregarding accidental degeneracy, we know that every set of degenerate energy levels forms a basis of (i.e. transforms according to) an irrep of the respective symmetry group:
	$$\mqty(\ket{E_i;1}\\\ket{E_i;2}\\\vdots\\\ket{E_i;d_i})\mbox{ forms a basis to }\Gamma^{(G)}_{r(i)}; \mqty(\ket{E^+_{i'};1}\\\ket{E^+_{i'};2}\\\vdots\\\ket{E^+_{i'};d^+_{i'}})\mbox{ forms a basis to }\Gamma^{\left(G_+\right)}_{r^+(i')}$$
	where $d_i, d^+_{i'}$ specify the total edegeneracies of the energy levels and $r,r^+$ are some surjective functions.
	We know (TODO cite this) that under a perturbation of a set of degenerate eigenstates, the distance of the new states from the subspace $\mathcal{D}^+$ spanned by the degenerate eigenstates goes like $O(n)$. Therefore, if $\ket{E_i;n}$ arises from perturbing a degenerate subspace of energy $E^+_j$, there exists an unnormalized vector $\ket{E^+_j;P}\in\mathcal{D}^+_j$ for which
	\begin{eqnarray*}
	\braket{E_i;n}{E^+_j;P}&=&1-O(\lambda)\\
	\qq{where} \braket{E^+_j;P}{E_i;n}&=&\braket{E^+_j;P'}{E_i;n}\forall \ket{E^+_j;P'}\in\mathcal{D}^+
	\end{eqnarray*}
	We can construct this state by projecting the perturbed eigenstate onto the original degenerate subspace:
	\begin{eqnarray*}
	\hat{P}^+_j&=&\sum_{m=1}^{d^+_j}\dyad{E^+_j;m}{E^+_j;m}\\
	\ket{E^+_j;P}&=&\hat{P}^+_j\ket{E_i;n}=\sum_{m=1}^{d^+_j}\ket{E^+_j;m}\braket{E^+_j;m}{E_i;n}
	\end{eqnarray*}
	The projection can be calculated directly for any choice of basis of $\mathcal{D}^+$, but we can simplify the calculation. Because $\braket{E_i;n}{E^+_j;P}\approx 1$, we expect both of these states to transform according to $\Gamma^{(G)}_{r(i)}$ under rotations in $G$. In general, the perturbation might reduce the degeneracy of $\mathcal{D}^+$ and render $d_i<d^+_j$ (even though it is also possible to have $d_i=d^+_j$, which is a special case). Then we can identify the subset of basis vectors of $\mathcal{D}^+$ which transform according to 	$\Gamma^{(G)}_{r(i)}$ by reducing $\Gamma^{(G_+)}_{r^+(j)}$ into $G$ and considering the basis vector surjection, as described above. Then we can approximate $\braket{E^+_j;m}{E_i;n}\approx 0$ for $m$ which are not surjectively mapped onto $\Gamma^{(G)}_{r(i)}$.
	
	Now we can express a general eigenstate of $\hat{H}$ as a superposition of an eigenstate of $\hat{H}^+$ with a norm that goes like $1-O(\lambda)$ and a remainder. We normalize the components and introduce superposition coefficients:
	$$\ket{E_i;n}=c_{ES}\ket{E^+_j;ES}+c_R\ket{R}$$
	where the \textit{elevated symmetry component} is defined as
	\begin{eqnarray*}
	\ket{E^+_j;ES}&=&\ket{E^+_j;P}\abs{\braket{E^+_j;P}{E^+_j;P}}^{-1/2}\\
	&=&\ket{E^+_j;P}\left(\sum_{m=1}^{d^+_j}\abs{\braket{E^+_j;m}{E_i;n}}^2\right)^{-1/2}\\
	c_{ES}&=&\braket{E^+_j;ES}{E_i;n}\\
	&=&\sqrt{\sum_{m=1}^{d^+_j}\abs{\braket{E^+_j;m}{E_i;n}}^2}\\
	&\sim& 1-O\left(\lambda^2\right)
	\end{eqnarray*}
	and the \textit{residual component} then becomes
	\begin{eqnarray*}
	\ket{R}&=&c_R^{-1}\left(\ket{E_i;n}-c_{ES}\ket{E^+_j;ES}\right)
	\end{eqnarray*}
	By demanding norm 1, we obtain:
	\begin{eqnarray*}
	1&=&c_R^{-2}\left(\braket{E_i;n}{E_i;n}+c_{ES}^2\braket{E^+_j;ES}{E^+_j;ES}-2c_{ES}\braket{E_i;n}{E^+_j;ES}\right)\\
	c_R&=&\sqrt{1-c_{ES}^2}\\
	&\sim & O(\lambda)\\
	\ket{R}&=& \left(1-c_{ES}^2\right)^{-1/2}\left(\ket{E_i;n}-c_{ES}\ket{E^+_j;ES}\right)
	\end{eqnarray*}
	
	\textit{A note about the ambiguity of construction of $\hat{H}_+$}. We see that the construction of $\hat{H}_+$ is not unique, and indeed there may be multiple options for the general shape of $\hat{H}_+$. It is important to note that we need not consider all options of $\hat{H}_+$ as if their effects were cumulative--the formulation of $\hat{H}$ as a perturbation of $\hat{H}_+$ should retrieve the same physics regardless of $\hat{H}_+$ should an infinite-order (no approximation) perturbation theory be used. However, the error associated with a 1st order perturbation analysis is inversely proportional to $\braket{E_i;n}{E_j^+;P}$, and thus a choice of $\hat{H}_+$ which maximises the norm of the elevated symmetry component will be the most precise one. Then, if a certain symmetry elevation suppresses a certain transition, then the symmetry suppression factor for a particular choice of $\hat{H}_+$ is in general a lower bound on the true symmetry suppression factor, as there might exist a different choice of $\hat{H}_+$ with a higher norm of the elevated symmetry component.
	
	\subsection{Evaluation of transition rates}
	
	
	
	\begin{thebibliography}{10}

\bibitem{karlsson}
Karlsson, K. F. \textit{et al} (2015), Spectral signatures of high-symmetry quantum dots and effects of symmetry breaking. \textit{New Journal of Physics}, \textbf{17} 103017

\bibitem{altmann}
Altmann, S. L., Herzig, P. (1994), \textit{Point-Group Theory Tables}. Oxford: Clarendon Press


\end{thebibliography}	
	
\end{document}